\subsection{جمع‌بندی و نتیجه‌گیری}


%introduction
در این مستند، گزارش مختصری درباره مساله تولید خودکار شرح بر تصاویر و روش‌های پیشنهادی برای حل چالش‌های موجود در این مسیر را ارائه دادیم. مساله تولید خودکار شرح بر تصاویر، به معنای تولید جملات زبان طبیعی برای هر تصویر است به طوری‌که این جملات شامل سه شرط زیر باشند:
\begin{enumerate}
\item صحنه، اجسام موجود در تصویر، رابطه مکانی اجسام و اطلاعاتی از این دست که به درک تصویر کمک می‌کنند باید در جملات تولید شده وجود داشته باشند و دقیق و کامل باشند.
\item جملات تولید شده، خود، باید به لحاظ معنایی، دستور زبان و املایی صحیح بوده و نقصی نداشته باشند.\
\item جملات تولید شده باید با تصاویر مرتبط با خود، سازگاری داشته باشند.
\end{enumerate}

ایده‌های اولیه در این مسیر از پژوهش‌های موجود در زمینه ترجمه ماشین ایجاد شده است که در آن‌ها، ابتدا یک جمله ورودی از یک زبان مبدا، با استفاده از روش‌های مختلفی به یک بردار ویژگی تبدیل می‌شود و سپس در مرحله دوم، بردار ویژگی حاصل، با استفاده از روش‌های خاص دیگری به جملات زبان طبیعی به زبان مقصد، تبدیل می‌شوند. حال اگر به جای جمله از زبان مبدا، یک تصویر به این سامانه وارد شود و با روشی بتوان این تصویر را به همان بردار ویژگی نگاشت کرد، جمله نهایی معادل معنایی تصویر ورودی خواهد شد. با استفاده از این فرایند می‌توان به طور خودکار برای تصاویر، شرح مناسبی به زبان طبیعی تولید کرد.
\\
در این مسیر، دو چالش عمده در پیش رو وجود دارد که باید مرتفع شوند:

\begin{enumerate}
\item چالش درک صحنه
\\
فرایند استخراج اطلاعات بصری نهفته در تصویر و بازنمایی مناسب این اطلاعات را به گونه‌ای که بتواند برای پردازش‌های بعدی مناسب باشد، فرایند درک صحنه می‌نامند. روش‌های مختلف و متعددی برای حل این چالش، تاکنون مطرح شده‌اند. در این مساله‌، باید بتوان تصاویر ورودی را به نحوی موثر و مفید به فضای ویژگی‌ها نگاشت کرد به طوری‌که بازنمایی حاصل، بتواند در مرحله تولید جمله، منجر به تولید جملات معنادار و مناسب شود.
\item چالش تولید جمله
\\
تولید جملات به زبان طبیعی که علاوه بر صحت معنایی، دستور زبانی و املایی، قادر به توصیف و تفسیر اطلاعات غیر قابل تفسیر برای کاربران انسانی هستند، از جمله مهم‌ترین و پویاترین حوزه‌های پژوهشی در زمینه هوش مصنوعی است و توجه پژوهش‌گران بسیاری را به‌ خود جلب کرده است. در این مساله، باید بتوان بردار ویژگی حاصل از تصویر را که در مرحله درک صحنه تولید شده است، به نحوی کارا و موثر به یک جمله در زبان طبیعی نگاشت کرد.
\end{enumerate}


% scene understanding
\subsection{درک صحنه}
اولین مرحله از فرایند تولید خودکار شرح برای تصاویر، مرحله درک صحنه است. در این مرحله، تصاویر ورودی تحت عملیات مختلفی به فضای معنایی نگاشت می‌شوند. فضای معنایی در این‌جا، می‌تواند فضای شامل میدان‌های اطلاعاتی از پیش تعیین شده (مانند فضای سه‌تایی‌های «جسم، رخداد، صحنه») یا فضای بردار ویژگی‌ها باشد.
\\
روش‌های مختلفی برای نگاشت تصویر ورودی به فضای معنایی ارائه شده است که به طور کلی می‌توان عموم آن‌ها را به دو بخش تقسیم کرد:
\begin{enumerate}
\item
روش‌های مبتنی بر مدل‌های گرافی احتمالاتی\\
در این روش‌ها با استفاده از مدل‌های استاندارد گرافی احتمالاتی موجود یا با ارائه یک مدل گرافی احتمالاتی، تصویر ورودی به فضای معنایی نگاشت می‌شود. در روش‌های مبتنی بر این مدل‌ها، با ارائه یک توزیع احتمال برای نقاط مختلف در فضای معنایی، محتمل‌ترین نقطه برای تصویر به عنوان نقطه نظیر تصویر، انتخاب می‌شود.
	\begin{enumerate}
		\item مدل میدان تصادفی مارکف\\

یک نمونه از روش‌های مبتنی بر مدل میدان تصادفی مارکف که برای درک صحنه از آن استفاده شده است، در پژوهش\cite{Farhadi2010every} 
ارائه شده است. درک صحنه در این پژوهش با ارائه یک سه‌تایی «جسم، فعالیت، صحنه» به‌ازای هر تصویر، تعریف شده است. مبتنی بر همین تعریف، یک مدل میدان تصادفی مارکف شامل سه گره که دو‌به‌دو به هم متصل هستند، تعریف شده است. هر یک از گره‌های موجود در این مدل، نماینده یکی از میدان‌های سه‌گانه تعریف شده در فضای معنایی هستند. با تعریف توابع پتانسیل مختلف روی هر گره و توابع پتانسیل مختلف روی هر یال، یک تابع توزیع توام برای تمام متغیرهای تصادفی موجود در مدل ارائه شده است.
\\
با محاسبه مقادیر پتانسیل برای تصاویر مختلف موجود در مجموعه‌آموزشی و با استفاده از یک ماشین بردار پشتیبان، بردارهای ویژگی شاخص برای هر گره محاسبه می‌شوند. از این بردارهای ویژگی بعدا برای انطباق تصاویر با مقادیر مختلف در هر گره استفاده می‌شود.
\\
در این پژوهش،‌ با یافتن نزدیک‌ترین همسایه‌های یک تصویر بر حسب معیار شباهت با بردارهای ویژگی شاخص  و میانگین‌گیری روی مقادیر هر گره، بهترین انطباق تصویر و نقاط فضای معنایی به‌دست می‌آید. به این ترتیب، برای هر تصویر ورودی، می‌توان نقطه نظیر در فضای معنایی را مشخص کرد.
		
		\item مدل میدان تصادفی شرطی\\
		در پژوهش\cite{fidler2013sentence} 
		یک مدل میدان تصادفی شرطی سلسله‌مراتبی برای درک صحنه ارائه شده است که شامل دو سطح انتزاع است. برای گره‌های موجود در هریک از سطوح انتزاع مدل، یک دسته متغیر تصادفی تعریف شده و برای کل مدل سه نوع تابع پتانسیل مختلف معرفی شده است.
		\\
		اولین دسته از توابع پتانسیل معرفی شده در این بخش، توابع پتانسیل قطعه‌بندی یگانی هستند که به منظور یکپارچه‌سازی نقاط داخل یک قطعه تعریف شده‌اند. توابع پتانسیل دیگری برای انطباق بین متغیرهای تصادفی موجود در بین دو سطح انتزاع تعریف شده‌اند که در صورت مغایرت مقادیر اختصاص داده‌شده به متغیرهای موجود بین دو سطح، مقدار $-\lambda$ و در غیر این صورت مقدار صفر دارند. این توابع در شرایطی که مقادیر متغیرهای موجود در دو سطح با هم یکسان نباشد، یک مقدار جریمه به تابع هدف اضافه می‌کنند. آخرین دسته از توابع پتانسیل مورد استفاده، برای انطباق تصویر با دسته تشخیص داده‌شده اجسام تعریف شده است که توسط فلزنسوالب ارائه شده و به روش دی پی ام مشهور است.
		
		\item سایر مدل‌های گرافی احتمالی
در پژوهش\cite{li2007and}، یک مدل گرافی احتمالی مولد برای نگاشت تصویر به فضای معنایی ارائه شده است. در این مدل، از دو سطح تصویر استفاده شده است؛ تصویر سطح جسم و تصویر سطح صحنه. برای تصویر سطح صحنه، یک متغیر تصادفی، بیان‌کننده دسته صحنه و برای تصویر سطح جسم دو متغیر تصادفی، بیان‌کننده دسته و شکل جسم، ارائه شده است. روابط بین متغیرهای تصادفی در این پژوهش، براساس نحوه تولید متغیرهای تصادفی و روابط منطقی موجود بین آن‌ها طراحی شده‌اند.
\\
تصویر ورودی در این پژوهش، ابتدا به نواحی کوچک 10*10 تقسیم می‌شود و مطابق با روش توضیح داده شده، مقدار توابع پتانسیل مختلف برای هر کدام از متغیرهای تصادفی، در هر ناحیه، محاسبه می‌شود. در این پژوهش، یک تابع احتمال شرطی برای متغیرهای تصادفی ارائه شده است که در مرحله استنتاج، با استفاده از روش تخمین بیشترین احتمال، برچسب‌های هر تصویر مشخص می‌شوند.
	\end{enumerate}

\item
روش‌های مبتنی بر استفاده از شبکه‌های عصبی کانولوشنی عمیق\\
در این روش‌ها، با ارائه یک شبکه عصبی کانولوشنی عمیق و تعریف کردن تابع هدف برای شبکه، تابع نگاشت تصویر و فضای معنا تشکیل می‌شود. پس از ارائه توابع هدف برای هر شبکه، با بهینه‌سازی آن تابع، پارامترهای موجود در شبکه آموزش داده می‌شوند.
\\
در پژوهش\cite{Girshick_2014_CVPR}، روشی ارائه شده است که طی آن یک تصویر، به نواحی کوچک‌تر تقسیم می‌شود به طوری‌که هر ناحیه به‌وجودآمده، به طور یکپارچه، حاوی یک جسم باشد و هر جسم تنها در یک ناحیه قرار بگیرد. این روش موسوم به روش \lr{RCNN} است. در این روش، دو ویژگی برای یک ناحیه‌بندی خوب در تصاویر ارائه شده است و پیرو این ویژگی‌ها، روشی برای طرح نواحی پیشنهادی در یک تصویر که دارای این دو ویژگی باشد، ارائه شده است.
\\
ویژگی مطرح شده اول برای ناحیه‌بندی تصاویر این است که، ناحیه‌های ایجاد شده در هر تصویر، می‌توانند در ابعاد مختلف وجود داشته باشند زیرا اجسام موجود در تصاویر، ممکن است اندازه و تعداد متفاوتی داشته باشند. دومین ویژگی برای یک ناحیه‌بندی خوب، این است که معیار انتخاب نواحی نباید برای تمام تصاویر، یکسان در نظر گرفته شود؛ زیرا معیارهای مختلف برای ناحیه‌بندی تصاویر در شرایط مختلف، رفتارهای متفاوتی از خود نشان می‌دهند. بنابراین باید از معیارهای مختلف برای تعیین نواحی استفاده نمود.
\\
در این پژوهش، ابتدا تصاویر مطابق با یک معیار اولیه، به مجموعه‌ای از نواحی اولیه تقسیم می‌شوند. سپس با استفاده از معیارهای مختلف مانند فضاهای رنگی مختلف،‌ معیارهای شباهت مختلف و نقاط اولیه متفاوت، با پیروی از یک روش حریصانه، نواحی کوچکتر که به یک‌دیگر شبیه‌تر هستند با هم ترکیب شده و نواحی بزرگتر را می‌سازند. نواحی ایجاد شده در این روش، سپس به یک شبکه عصبی کانولوشنی عمیق داده می‌شوند و برای هر ناحیه، یک بردار ویژگی 4096 بعدی ایجاد می‌شود که هر ناحیه با آن بازنمایی شود.
\\
در پژوهش \cite{karpathy2014deep} با استفاده از روش \lr{RCNN} و تعریف دو تابع هدف دیگر برای شبکه، روشی ارائه شده است که طی آن بتوان تصاویر و جملات را به طور دوطرفه به یک‌دیگر نگاشت کرد. توابع هدف تعریف شده در این پژوهش، دو تابع مختلف هستند. اولین تابع هدف، یک تابع هدف سراسری است. این تابع به این منظور تعریف شده است که تصاویر و جملاتی که مطابق با محاسبات شبکه عصبی ارائه شده، بیشترین شباهت را با یک‌دیگر دارند، در واقعیت هم شبیه‌ترین تصاویر و جملات به یک‌دیگر باشند. تابع هدف دوم برای این شبکه به این شکل تعریف شده است که نواحی استخراج شده از تصویر و عبارات استخراج شده از جملات که در روش ارائه شده، بیشترین شباهت را به یک‌دیگر دارند،‌ در واقعیت هم بیشترین شباهت و ارتباط را با یک‌دیگر داشته باشند.
\\
در این پژوهش، تصاویر ورودی با استفاده از روش \lr{RCNN} به نواحی مختلف تقسیم شده و ۱۹ ناحیه با بیشترین اطمینان از بین این نواحی انتخاب می‌شود. این ۱۹ ناحیه به همراه خود تصویر به عنوان ۲۰ تصویر مختلف مورد استفاده قرار می‌گیرند. جملات ورودی با استفاده از روشی که در فصل تولید جملات زبان طبیعی توضیح داده خواهد شد، به عبارات مختلف تقسیم می‌شوند و بین هر عبارت استخراج شده و هر یک از ۲۰ تصویر موجود، یک معیار شباهت محاسبه شده و بیشترین شباهت‌ها با هم درنظر گرفته می‌شوند. معیار شباهت مورد استفاده در این روش، ضرب داخلی بین بردارهای ویژگی عبارات و نواحی است. عبارات و نواحی که بیشترین شباهت را با یک‌دیگر دارند برای تولید جمله به مرحله بعد، ارسال می‌شوند.
\end{enumerate}

% sentence generation
\subsection{تولید جمله}
چالش تولید جمله یکی از قدیمی‌ترین و پویاترین حوزه‌های فعالیتی و پژوهشی در هوش مصنوعی است که از اواسط قدن بیستم، توجه پژوهش‌گران بسیاری را به خود جلب کرده است. روش‌های مختلفی برای حل این مساله ارائه شده‌اند. از جمله این روش‌ها می‌توان به موارد زیر اشاره کرد:

\begin{enumerate}
\item تولید زبان طبیعی
\\
در این دسته از روش‌ها که از اواخر دهه بیستم تا کنون مورد استفاده قرار می‌گیرند، با طی فرایند در یک چارچوب کلی، سعی در تولید جملات مناسب دارند. این دسته از روش‌ها عموما برای تفسیر خودکار داده‌هایی که برای کاربران انسانی غیر قابل تفسیر هستند یا تفسیر دشواری دارند، به‌کار می‌روند. در این روش‌ها ابتدا با استفاده از ویژگی‌های مختلفی که در داده‌های ماشینی (داده‌های قابل تفسیر برای ماشین) کلمات مناسب انتخاب شده و سپس با استفاده از کلمات منتخب، عبارات زبانی (با جایگشت دادن کلمات و حذف عبارات غیر محتمل) تولید می‌شوند. سپس با اعمال قواعد دستور زبان و چینش عبارات زبانی در کنارهم، جملات نهایی تولید می‌شوند.
\item نزدیک‌ترین همسایه
\\
در این دسته از روش‌ها سعی می‌شود با ورود یک تصویر و نگاشت آن به فضای ویژگی‌ها، جمله‌ای از میان تمام جملات موجود در مجموعه‌داده انتخاب شود که بیشترین مشابهت با بردار ویژگی تصویر را دارد. بزرگ‌ترین مشکل در این روش‌ها انتخاب معیار مناسب برای محاسبه فاصله بین یک جمله و بردار ویژگی حاصل از تصویر است. در این روش، علاوه بر این‌که نیاز به وجود مجموعه‌داده وسیع و پوشا وجود دارد، ممکن است جمله نهایی، در انتها گویا و بیان‌کننده تمام جوانب تصویر نباشد و یا حتی با تصویر ورودی سازگاری نداشته باشد.
\\
برای حل این مشکل، سعی شد به جای استخراج نزدیک‌ترین جمله به تصویر موجود، مشابه‌ترین عبارات زبانی را با شکستن جملات موجود به عبارات سازنده، انتخاب کرده و با بهره‌گیری از روش تولید زبان طبیعی و یا روش‌های دیگر، چینش مناسبی از این عبارات را که در قالب یک یا چند جمله بیان شوند، تولید و به عنوان شرح بر تصویر، نمایش داد.
\item استفاده از قالب‌های زبانی آماده
\\
با وجود فعالیت‌های گوناگون در این زمینه و استفاده از روش‌های مختلف، هم‌چنان تضمین صحت جمله خروجی، کار دشواری است. به همین دلیل، سعی شد با ارائه یک یا چند قالب زبانی آماده و از پیش تعیین شده برای جملات، مانند قالب‌های جملات خبری، صحت جملات نهایی را تضمین کرد. در این دسته از روش‌ها، ویژگی‌های مختلفی از تصویر استخراج می‌شود که هریک از این ویژگی‌ها یا همه آن‌ها در کنار هم قادر هستند نقش‌هایی مانند «فعل»، «فاعل»، «مفعول» و موارد مشابه را در جمله متناظر با تصویر مشخص کنند. با استخراج کلمات مناسب و شناخت نقش آن‌ها در جمله و جای‌گذاری هر یک از این کلمات در مکان مناسب نقشی خود در قالب از پیش تعیین شده، جمله متناظر با هر تصویر استخراج می‌شود.
\item استفاده از شبکه‌های عصبی بازگشتی
\\
اگر چه استفاده از قالب‌های آماده و از پیش تعیین شده، تا حدی مشکلات موجود را حل می‌کند اما هم‌چنان چالش بزرگ‌تری حل نشده باقی مانده است. تولید جملات جدید، استفاده از کلمات و عبارات جدید و ابتکاری به طوری‌که علاوه بر تضمین رعایت دستور زبان، بتوان معنای جمله را نیز متضمن شد، چالش بزرگی است که در این مسیر کماکان وجود دارد.
\\
استفاده از شبکه‌های عصبی بازگشتی یکی از بهترین راه‌کارهای موجود برای حل این مشکل و رویارویی با این چالش هستند. استفاده از این شبکه‌ها در اواخر قرن بیستم در بین پژوهش‌گران رواج پیدا کرد تا جایی که ناپایداری الگوریتم پس‌انتشار خطا در آموزش این شبکه، راه را برای پژوهش‌های بعدی بست. پس از ارائه یک روش مناسب برای بهینه‌سازی بدون هسین در سال 2010، روشی برای آموزش یک شبکه عصبی بازگشتی موسوم به شبکه عصبی بازگشتی ضربی بر مبنای بهینه‌سازهای بدون هسین ارائه شد و نتایج آن به طور چشم‌گیری از روش‌های موجود بیشتر بود.
\\
ارائه شبکه عصبی بازگشتی ضربی، نقطه عطفی در مسیر علم در راستای حل چالش تولید جمله به حساب می‌آید. از حدود سال 2011 به بعد، استفاده از شبکه‌های عصبی بازگشتی برای تولید جمله به پویاترین و پرفعالیت‌ترین حوزه در مسائل مربوط به تولید جمله، به حساب می‌آید.
\end{enumerate}

خانم لی و همکارانش در سال 2015، در پژوهش \cite{karpathy2015deep}، با استفاده از شبکه‌های عصبی کانولوشنی عمیق و دو نوع از شبکه‌های عصبی بازگشتی موسوم به شبکه‌های عصبی بازگشتی مالتی‌مودال و شبکه‌های عصبی بازگشتی دوطرفه، روش مناسبی برای تولید خودکار شرح بر تصاویر ارائه داده است.
\\
در این پژوهش، ابتدا با بهره‌گیری از روش شبکه عصبی کانولوشنی ناحیه‌ای، نواحی از تصویر که شامل تصویر اجسام است، استخراج شده و با استفاده از یک شبکه عصبی کریشفسکی، بردار ویژگی برای هر ناحیه محاسبه می‌شود. سپس با بهره‌گیری از یک شبکه عصبی بازگشتی دوطرفه، عبارات مختلف از جمله استخراج و بردارهای ویژگی برای هر عبارت محاسبه می‌شود. سپس با استفاده از یک تابع هدف و مدل میدان تصادفی مارکف، هم‌ترازسازی بین نواحی و عبارات زبانی صورت گرفته و مدل آموزش داده می‌شود.
\\
 در ادامه با تخمین بهینه پارامترهای موجود و با استفاده از شبکه عصبی بازگشتی مالتی‌مودال، توزیع احتمال بهترین کلمه بعدی در یک جمله با داشتن کلمات قبلی و محتوای حاصل از بردار ویژگی محاسبه شده روی نواحی تصویر، محاسبه شده و بهترین کلمه بعدی تولید می‌شود. این کار تا جایی ادامه می‌یابد که شبکه، نشانه مخصوص پایان جمله را تولید کند.
  