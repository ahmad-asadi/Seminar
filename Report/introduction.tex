\subsection{مقدمه}

به دنبال پیشرفت تکنولوژی در ساخت دوربین‌های عکاسی و ورود دوربین‌های نیمه‌خودکار و خودکار به بازار، تعداد زیادی از کاربران سیستم‌های رایانه‌ای به استفاده از این تکنولوژی در ثبت تصاویر مورد علاقه خود جذب شده‌اند. دقت و کیفیت مطلوب تصویربرداری از یک سو و سهولت استفاده از دوربین‌ از سوی دیگر، باعث شده‌اند تعداد تصاویر ثبت شده توسط کاربران به طور روزافزون افزایش یابد؛ به‌طوری‌که امروزه اغلب کاربران، تعداد بی‌شماری از این تصاویر را در گوشی‌های تلفن همراه، تبلت‌ها و رایانه‌های شخصی خود نگه‌داری می‌کنند.
از جمله مشکلاتی که در اثر ایجاد این حجم وسیع از تصاویر بوجود آمده، مشکل مدیریت این تصاویر و یافتن تصاویر خاص بین مجموعه بزرگی از تصاویر موجود، است.
\\
برای دست‌یابی به سامانه‌ای که بتواند تعداد زیادی از تصاویر موجود را مدیریت نماید، ابتدا باید صحنه موجود در تصویر را به درستی درک کرد. درک صحیح از صحنه، عبارت است از بیان تصویر به نحوی که اطلاعات کلی موجود و هدف اصلی تصویر، واضح و مشخص باشد. این بیان می‌تواند شامل اجسام موجود در تصویر، رابطه مکانی بین اجسام، فعالیت به تصویر کشیده شده، شرایط محیطی موثر بر صحنه و مواردی از این دست باشد. از طرفی باید به نحوی محتوای تصاویر را بیان کرد که بتوان عملیات جستجو را بر اساس مدل بیان شده تصاویر انجام داد. در این‌صورت به‌ازای هر تصویر، یک نمونه از مدل مطابق با تصویر ایجاد و ذخیره خواهد شد. پرس‌و‌جوی\enfootnote{Query} کاربر، به فضای مدل نگاشت شده و تصویر معادل با مدلِ استخراج شده، به عنوان نتیجه جستجو نمایش داده می‌شود. علاوه بر این، مساله مدیریت تصاویر، به مساله مدیریت مدل‌های موجود کاهش داده می‌شود.
\\
تولید شرح کلی بر تصاویر\enfootnote{Holistic Image Caption}، بیان مناسبی از صحنه موجود در تصویر را ارائه می‌دهد. شرح تولید شده بر تصاویر، در قالب مجموعه‌ای از جملات زبان طبیعی\enfootnote{Natural Language Sentences} ارائه‌ می‌شود که عموما بیان‌گر اجسام موجود در صحنه‌، ارتباطات مکانی بین اجسام و اطلاعات مشخص دیگر است که در هر پژوهش می‌تواند متفاوت باشد. بنابراین، دست‌یابی به سامانه‌ای که قادر به تولید خودکار شرح کلی بر تصاویر باشد، اساسی‌ترین گام در راستای تولید نرم‌افزارهای مدیریت تصاویر است.
\\
یکی از اولین ایده‌های مطرح شده در این زمینه، با الهام از پژوهش­‌های صورت گرفته در زمینه ترجمه ماشین\enfootnote{Machine Translation}
 به‌وجود آمده‌ است که با هدف ترجمه جملات یک زبان به زبان دیگر به طور خودکار، انجام شده‌اند. در این راستا، یک جمله از زبان مبدا\enfootnote{Source Language}، با روش‌های مختلف تبدیل به یک بردار ویژگی\enfootnote{Feature Vector} می‌شود که مشخصه‌های اصلی جمله اولیه را نمایش می‌دهد. سپس بردار ویژگی حاصل با اعمال روش­‌های گوناگون دیگری، تبدیل به یک جمله از زبان مقصد\enfootnote{Destination Language} می­گردد که در آن تمام ویژگی‌های موجود در بردار ویژگی بیان شده‌‌اند.
با توجه به فرایند مذکور، اگر به جای جمله زبان مبدا، یک تصویر را به بردار ویژگی تبدیل و سپس با استفاده از روش‌های موجود قبلی، بردار ویژگی را به جمله زبان مقصد ترجمه نمود، جمله‌ای معادل با تصویر ورودی به‌دست خواهد آمد. که بیان‌گر محتوای به تصویر کشیده شده در تصویر ورودی است.
\\
شرح خودکار تصاویر، توجه پژوهش‌گران بسیار زیادی را به خود جلب کرده است و فعالیت‌های متنوع و متعددی در این راستا انجام شده است. علی‌رغم وجود پژوهش‌‌های فراوان و متفاوت، می‌توان یک بستر کلی برای تمام فعالیت‌های موجود در این زمینه ارائه داد. بر این مبنا، فرایند کلی که در عموم پژوهش‌های انجام‌شده، پی گرفته شده‌است، از دو بخش اساسی تشکیل می‌شود.
\begin{enumerate}
\item بازنمایی تصاویر، با استفاده از بردار ویژگی
\item تبدیل بردار ويژگی به‌دست‌آمده به جملات صحیح زبانی
\end{enumerate}


\subsection{تعریف مساله}
در این پروژه قصد داریم سامانه‌ای ارائه دهیم که قادر به تولید شرح کوتاه بر تصاویر باشد. دو دیدگاه اساسی در دست‌یابی به چنین سامانه‌ای مطرح است.
\begin{enumerate}
\item
 یافتن نقاط توجه\enfootnote{Attention Points} 
در تصاویر و تولید جملات توصیف‌کننده اجسام مستقر در این نقاط به طوری‌که توصیف جسم مستقر در نقطه توجه و اجسام مرتبط با آن در جملات تولیدی، وجود داشته باشد.
\item  تولید شرح جامع بر تصاویر به طوری‌که تمام اجسام موجود در صحنه به همراه روابط موجود بین آن‌ها توصیف شوند. 
\end{enumerate}

شرح کوتاه تولید شده در این پروژه، به معنی تولید جملاتی است که مستقیما به توصیف صحنه، اجسام موجود در صحنه و روابط بین آن­ها می­پردازند.
به طور کلی، دو چالش عمده در این پژوهش مورد توجه قرار خواهد گرفت:
\begin{enumerate}
\item  توصیف صحنه باید دقیق باشد؛ به این معنی که اجسام موجود در صحنه باید به طور دقیق از هم تفکیک شده و دسته‌بندی شوند. تصویر توصیف شده باید در قالب مناسبی بازنمایی شود که بتوان به راحتی از آن برای تولید جمله استفاده نمود.
\item  جملات تولید شده برای شرح تصویر باید به لحاظ دستور زبان، املا و معنا صحیح بوده و با تصویر مرتبط خود سازگار باشند و آن را به درستی و دقت شرح دهند.
\end{enumerate}
